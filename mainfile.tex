\documentclass[12pt,letterpaper]{lsuetd}
\usepackage{setspace,graphics,dsfont,verbatim,paralist,indentfirst}
%bcb
\usepackage{graphicx}
\usepackage{subfig}
\usepackage{amsmath}
\usepackage{amssymb}
\usepackage{xspace}
\usepackage{booktabs}


%bibm
\usepackage{lipsum}
\usepackage{caption}
\usepackage{algorithm,algorithmic}
\usepackage{multicol}
\usepackage{array}
\usepackage{eqparbox}
%icdcs
\usepackage{url}
\usepackage{graphicx}
%\usepackage{subcaption}
\usepackage{xcolor}
\usepackage{pgfplotstable}
%\usepackage{filecontents}
\pgfplotsset{compat=1.16}
\usepackage{fancyhdr}
\usepackage{pifont}
\newcommand{\xmark}{\ding{55}}
\DeclareMathOperator{\EX}{\mathbb{E}}
\usepackage{listings}
\usepackage{array}
\definecolor{blue-main}{rgb}{0,0,1}
\definecolor{dkgreen}{rgb}{0,0.6,0}
\definecolor{gray}{rgb}{0.5,0.5,0.5}
\definecolor{mauve}{rgb}{0.58,0,0.82}
\lstset{ %
	language=C++,                % the language of the code
	basicstyle=\tiny,  % the size of the fonts that are used for the code
	showspaces=false,            % show spaces adding particular underscores
	showstringspaces=false,      % underline spaces within strings
	showtabs=false,              % show tabs within strings adding particular underscores
	frame=topline,               % adds a frame around the code
	tabsize=2,                   % sets default tabsize to 2 spaces
	captionpos=t,                % sets the cap%%%%%%%%%%%%%%%%%%%%%%%%%%%%%%%%%%tion-position to bottom
	breaklines=true,                % sets automatic line breaking
	breakatwhitespace=false,        % sets if automatic breaks should only happen at whitespace
	xleftmargin=\fboxsep,
	xrightmargin=-\fboxsep,
	firstnumber=1,
	title=\lstname,                   % show the filename of files included with \lstinputlisting;
	keywordstyle=\color{blue-main},          % keyword style
	commentstyle=\color{dkgreen},       % comment style
	stringstyle=\color{mauve},         % string literal style
	escapeinside={\%*}{*)},            % if you want to add a comment within your code
	morekeywords={*,...},               % if you want to add more keywords to the set
	escapechar={@},
	numbers=left,
	numbersep=5pt,                   % how far the line-numbers are from the code
	numberstyle=\tiny\color{gray}, % the style that is used for the line-numbers
	moredelim=**[is][\color{blue-main}]{^}{^}
}

\captionsetup[table]{labelsep=period}
\captionsetup[figure]{labelsep=period}

\usepackage{textcomp}
\usepackage{titlesec}
%\titleformat*{\section}{\normalsize\bfseries}
\titlelabel{\thetitle.\quad}
\usepackage{fourier}
\usepackage{pdfpages}
%HaRE
%\usepackage{moreverb}
%\usepackage[colorlinks,bookmarksopen,bookmarksnumbered,citecolor=red,urlcolor=red]{hyperref}
%\usepackage{floatrow}
%\usepackage{color}
%
\usepackage[strings]{underscore}
\usepackage{multirow}% http://ctan.org/pkg/multirow
\usepackage{hhline}
\usepackage{graphicx}

\setlength{\topmargin}{-0.5in}
\setlength{\textheight}{9.0in}
\addtolength{\evensidemargin}{-0.50in}
\addtolength{\oddsidemargin}{-0.50in}
\addtolength{\textwidth}{1.00in}
\setlength{\parindent}{1.75em}
\setlength{\parskip}{0ex}
\setcounter{tocdepth}{1}
\setcounter{secnumdepth}{4}
\setcounter{section}{1}

\newcommand{\todo}[1]{}
\renewcommand{\todo}[1]{{\color{red} TODO: {#1}}}

\newcommand\blfootnote[1]{%
  \begingroup
  \renewcommand\thefootnote{}\footnote{#1}%
  \addtocounter{footnote}{-1}%
  \endgroup
}

\makeatletter

\renewcommand\Huge{\@setfontsize\Huge{16pt}{18}} %%%%%%%%%%%%%%% Used for chapter titles
\renewcommand\huge{\@setfontsize\huge{14pt}{18}} %%%%%%%%%%%%%%% Used for chapter numbers
\renewcommand\Large{\@setfontsize\Large{12pt}{18}} %%%%%%%%%%%%%%% Used for section titles
\renewcommand\large{\@setfontsize\large{12pt}{18}} %%%%%%%%%%%%%%% Used for subsection titles
\renewcommand{\baselinestretch}{1.5}

\begin{document}
\renewcommand\@pnumwidth{1.55em}
\renewcommand\@tocrmarg{9.55em}
\renewcommand*\l@chapter{\@dottedtocline{0}{1.5em}{2.3em}}
\renewcommand*\l@figure{\@dottedtocline{1}{0em}{3.1em}}
\let\l@table\l@figure

\pagenumbering{roman}
\thispagestyle{empty}
\begin{center}
%The title page is first created.


{\Huge\bfseries
Optimizing the Performance of Multi-threaded Linear Algebra Libraries, a Task Granularity based Approach 
}\par

\vfill
\doublespacing
A Dissertation \\
\singlespacing
Submitted to the Graduate Faculty of the \\
Louisiana State University and \\
Agricultural and Mechanical College \\
in partial fulfillment of the \\
requirements for the degree of \\
Doctor of Philosophy \\
\doublespacing
in \\
                                       
Department of Electrical Engineering and Computer Science\\
\singlespacing
\vfill

by \\
Shahrzad Shirzad\\
B.Sc., Sharif University of Technology, 2006\\
M.Sc., K. N. Toosi University of Technology, 2009\\
December 2019

%If necessary, copy and paste the previous line here to include a master's degree.
\end{center}
\pagebreak
%The Copyright Page and Dedication sections can be added here, if desired.

%\chapter*{Copyright Page}
%\doublespacing
%\vspace{0.55ex}
%Insert the appropriate text for the copyright page here.
%\addcontentsline{toc}{chapter}{\hspace{-1.5em} {COPYRIGHT PAGE} \vspace{12pt}}
%\pagebreak

%\chapter*{Dedication}
%\doublespacing
%\vspace{0.55ex}
%Insert the appropriate text for the dedication or epigraph page here.  This part of the ETD must not exceed one page.
%\addcontentsline{toc}{chapter}{\hspace{-1.5em} {DEDICATION} \vspace{12pt}}
%\pagebreak

\chapter*{Acknowledgments}

%\vspace{0.55ex}


%The code below adds the Acknowledgments section to the Table of Contents.
\addcontentsline{toc}{chapter}{\hspace{-1.5em} {ACKNOWLEDGMENTS} \vspace{12pt}}
\pagebreak
%The Preface section can now be added, if desired.

%\chapter*{Preface}
%\doublespacing
%\vspace{0.55ex}
%Insert the appropriate text for the preface here.
%\addcontentsline{toc}{chapter}{\hspace{-1.5em} {PREFACE} \vspace{12pt}}
%\pagebreak
\setcounter{tocdepth}{1}
\singlespacing
\tableofcontents
\pagebreak

%The code below generates the List of Tables and adds it to the Table of Contents.
\renewcommand\@pnumwidth{1.55em}
\renewcommand\@tocrmarg{8.55em}
\addcontentsline{toc}{chapter}{\hspace{-1.5em} LIST OF TABLES \vspace{12pt}}
\listoftables
\pagebreak
%The code below generates the List of Figures and adds it to the Table of Contents.
\addcontentsline{toc}{chapter}{\hspace{-1.5em} LIST OF FIGURES \vspace{12pt}}
\listoffigures
\pagebreak
%The List of Nomenclature may be included here, if desired.

%\chapter*{List of Nomenclature}
%\doublespacing
%\vspace{0.55ex}
%Provide the definitions of the symbols used in your thesis or dissertation here.
%\addcontentsline{toc}{chapter}{\hspace{-1.5em} LIST OF NOMENCLATURE \vspace{12pt}}
%\pagebreak

%The code below adds the Abstract and places it within the Table of Contents.
\renewenvironment{abstract}{{\hspace{-2.2em} \huge \textbf{\abstractname}} \par}{\pagebreak}
\addcontentsline{toc}{chapter}{\hspace{-1.5em} ABSTRACT}
\begin{abstract}
\vspace{0.55ex}
\doublespacing

Linear algebra libraries play a very important role in HPC applications. In this work we propose a method to tune the performance of a linear algebra library based on a set of compile-time and runtime characteristics including the machine architecture, the expression being evaluated, the number of cores to run the application on, the type of the operation, and also the size of the matrices, to be able to get as close as possible to the highest performance. 

There has been an extensive amount of work and study done on how to optimize the compute kernels for matrix-matrix multiplication. In this thesis, we are interested in all types of operations, and our focus is on machine learning applications, where we are potentially dealing with very large matrices and creating temporaries could be very expensive. 

For this purpose we decided to use Blaze C++ library, a high performance template-based math library that gives us this option to access the expression tree at compile time, along with HPX, a C++ standard library for concurrency and parallelism, as our runtime system.
We propose here that instead of dividing the work equally among the cores and assigning one chunk to each core, we should be able to achieve a higher performance by selecting the right amount of work to be assigned to one core, based on a set of runtime and compile-time parameters.  

Studying the significant amount of data we collected with different configurations, we concluded that, grain size, which is the amount of work assigned to each task, is the key factor for the performance when number of cores is fixed. With this assumption, we tried two different approaches to model the relationship between performance and grain size, in order to find a range of grain size that could lead us to the maximum performance. 

First model uses a polynomial function to fit the data both for throughput vs. grain size, and throughput vs. number of cores. On the other hand, the second model was developed by studying the behavior of throughput vs. grain size , and integrating it with the an extension of the Universal Scalibility Law(USL). This developed model fitted our data reasonably and could be very important in further understanding of the parallelism. 

The primary results suggest that, using the data collected for matrix-matrix addition, we are able to improve the performance for matrix-matrix addition, by finding the right range of grain size.

Having the mentioned models, we changed the current implementation of the HPX backend for Blaze by adding two parameters to represent the unit of work and the number of these units included in each task, for fine grained control of the parallelism, which is possible through HPX runtime system.
Also, a complexity estimation function has been added to Blaze as an estimate of the number of floating point operations occurring in each unit of work. 

Our data was only limited to one specific operation, matrix-matrix addition for two raw-major matrices, one matrix size at a time. In the next step, this problem should be generalized to different matrix sizes, architectures, and arbitrary complex expressions. 

\end{abstract}

\pagenumbering{arabic}
\addtocontents{toc}{\vspace{12pt} \hspace{-1.8em} CHAPTER \vspace{-1em}}
\singlespacing
\setlength{\textfloatsep}{12pt plus 2pt minus 2pt}
\setlength{\intextsep}{6pt plus 2pt minus 2pt}
\chapter{Introduction}\label{Introduction}
\doublespacing
The current microprocessors are able to deliver a peak performance in range of hundreds of Mflops to Gflops. But in order to achieve these performances a lot of effort needs to be made to optimize your program based on the architecture it would be run on\cite{whaley1998automatically}. 

The core element of many high performance computing applications is the linear algebra library. linear algebra libraries like ATLAS, SPIRAL,... try to use hardware-specific optimizations to improve their performance. In this work, we are trying to optimize the performance based on the application parameters such as matrix size, operation, the expression, data layout, and also the machine architecture.   


\vspace{\baselineskip}
\section{Thesis Statement}
The main objective of this thesis is to propose a hybrid runtime and compile-time solution for a linear algebra library to fully take advantage of the available parallelism and resources. 

We chose Blaze math library since it is a nice high performance template-based C++ library that allows you to access the expression tree for each assignment at compile time, and we chose HPX as an asynchronous may-task runtime system to manage the parallelism. 

HPX makes it possible to create thousands to millions of lightweight user threads, to avoid expensive context switching. On the other hand, although the overhead of creating one task is negligible, creating millions of tasks when the execution time of the program itself is small, could become significant and cause performance degradation. On the other hand if we create too few tasks it would be very likely for us to not use our resources properly. So in very application in many task systems, it's very important to chose the amount of work assigned to each task, called grain size, properly. 

Through analyzing and modeling the relationship between throughput and grain size, we would be able to identify a range of grain size that leads us to maximum performance. Once decided how big one unit of work should be, based on the identified range we would be able to decide on how many units of work should be packed into one task.

On the other hand, there are different models to express the relationship between the throughput and the number of cores. Here, we are interested in developing a model to be as realistic as possible to imitate the behavior of the throughput against both grain size and number of cores. This would help us find how to manage the parallelism in our system to achieve the highest performance possible.
 
\vspace{\baselineskip}
\section{Contributions}
There has been a wide study, mostly by Gunther\cite{gunther2000practical,gunther2002new,gunther2007guerrilla,gunther2011new}, on different models to represent the relationship between the throughput and the number of cores, for a fixed size problem. 
Grubel et.al\cite{grubel2015performance} has studied the effect of task granularity on the performance with a fixed number of cores. 
Our contributions could be summarized into:
\begin{itemize}
	\item{We propose a novel physical model to represent how the execution time is expected to change based on grain size.}
	\item{To our knowledge, there has not been a work to create a 3D model of the throughput, grain size, and number of cores.} 
	\item{We are proposing a method to apply the developed model to a linear algebra library, in a way specific to our application, and the machine architecture.}
\end{itemize}
 

\section{Document Organization}
In Chapter~\ref{Background}, we will explain briefly the background needed for this thesis, including Blaze and HPX library, the effect of task granularity, and the Universal Scaling Law(USL) method for modeling the throughput based on number of cores.  
Chapter~\ref{Literature} refers to other works that have been done in our area of our focus. We explain our proposed method to optimize the performance, along with the models we used in Chapter~\ref{Method}. Our work heavily relies on the collected data, the environment we collected the data from and also the library versions are mentioned in Chapter~\ref{Results}.
Finally we discuss our concerns and the further steps that needs to be taken in Chapter~\ref{Future}.

\vspace{\baselineskip}

\pagebreak
\singlespacing

\chapter{Background}\label{Background}
\doublespacing
\subsection{Experiments}
In order to capture the relationship between number of cores, chunk\textunderscore size, block\textunderscore size, and the performance, we ran a series of experiments with different values for first three parameters and measured the number of floating point operations per second performed. 

For these experiments ,at the first step we selected the \textit{DMatDMatADD} benchmark which was implemented in Blazemark. \textit{DMatDMatADD} benchmark is a level 3 BLAS function to perform matrix-matrix addition in the form of $A=B+C$, where $A$, $B$, $C$ are square matrices of the same size. 

To avoid adding the scheduling overhead for small matrix sizes, Blaze uses a threshold to start parallelization, which is specific to the type of operation. For matrix-matrix addition, if the number of elements in the matrix is greater than 36100 elements(which is equivalent to a square matrix size of size 190 by 190) Blaze uses the configured backend to parallelize the assignment operation. For this reason, we start our experiments with matrix size of 200x200 and gradually increase the size to 1024x1024. 

  

\subsection{L2 cache miss analysis}
In this set of experiments we used the performance counters integrated into HPX to measure the cache misse rate for different grain\textunderscore sizes w

-grain\textunderscore size

\pagebreak
\singlespacing

\chapter{Literature Review}\label{Literature}
\doublespacing
\section{Literature Review}

Loop scheduling techniques has been extensively studied by different researchers. In \cite{donfack2012hybrid} the authors propose a hybrid static/dynamic method for loop scheduling that improves the performance of dense matrix factorization, compared to both fully static and fully dynamic scheduling. The authors of \cite{donfack2012hybrid}, divide the dependency graph into two subgraphs, one of which is scheduled dynamically and the other one is scheduled statically. The tasks on the critical path are scheduled statically and each thread is forced to prioritize the static tasks\cite{donfack2012hybrid}. They were able to improve data locality and scheduling overhead, while creating a more balanced workload. 

%\cite{xue2007locality}
% \cite{tang1986processor},\cite{polychronopoulos1987guided},\cite{hummel1992factoring},\cite{kruskal1985allocating}
%	

The previous work on predicting the performance of a parallel application mainly focuses on three major types of models: analytical, trace-based, and empirical models\cite{malakar2018benchmarking}. 

The analytical models\cite{blagojevic2008modeling},\cite{kerbyson2001predictive},\cite{valiant1990bridging}, while providing an arithmetic formula to represent the execution time of an application, require a deep understanding of the application, to apply platform-specific optimizations, and can not be generalized to different domains and architectures\cite{lee2007methods},\cite{sun2017automated},\cite{pllana2007performance}.
Traced-based models, on the other hand, use the traces collected through instrumentation, to predict the performance. These models, opposed to analytical models, do not rely on an expert's knowledge of the application, but while adding some overhead to the runtime, these models require a large storage space to save the traces, and are hard to interpret\cite{sun2017automated}.   
In empirical modeling, the results obtained from running an application with a set of parameters on a specific set of machines to build a model for unknown set of application and system parameters\cite{malakar2018benchmarking}. This type of modeling includes machine learning based approaches.

In \cite{ipek2005approach}, the authors use neural networks to predict the performance focusing on SMG2000 application, a parallel multigrid solver for linear systems\cite{falgout2002hypre}, on two different platforms. Defining application parameters $N_x$, $N_y$, $N_z$, representing the working set size per processor, and $P_x$, $P_y$, $P_z$, describing the three-dimension processor topology, as the features, \cite{ipek2005approach} uses a fully connected neural network to learn the model. Since they use absolute mean square error as the loss function, they use stratification to replicate samples with lower values by a factor which is proportional to their target value. They also apply bagging technique to decrease the variance in the model. As they increase the size of the training set to 5K points, they reach an error rate of 4.9\%. 

As a trace-based model, \cite{sun2017automated} analyzes the abstract syntax tree of the code and collects data through inserting special code for instrumentation when encounters 4 different situations, namely, assignments, branches, loops, and MPI communications. The authors then use 5 different machine learning methods including random forests, support vector machine, and ridge regression to build a prediction model from the collected data. Through applying two filtration processes, they were able to decrease the amount of overhead introduced along with the storage space requirement. Their results were inclined towards random forest, mainly because of the lower impact of categorical features on it, which is helpful in general cases where we do not have any knowledge about the type of features\cite{sun2017automated}.  
	
In \cite{malakar2018benchmarking} the authors investigate a set of machine learning techniques, including deep neural networks, support vector machine, decision tree, random forest, and k-nearest neighbor to predict the execution time of 4 different applications. Each of these applications require a certain set of features as input, for example, for the miniMD application in molecular dynamics, the number of processes and the number of atoms were considered as the input features, while for miniAMR, an application for studying adaptive mesh refinement, number of processes and also block sizes in $x$, $y$, and $z$ direction, where used as the input features. While achieving promising results especially for deep neural networks, bagging, and boosting methods, \cite{malakar2018benchmarking} suggest utilizing transfer learning through deep neural networks to predict performance on other platforms.
%	
%\cite{pusukuri2011thread}
%\cite{marin2004cross}

Although concentrating on GPUs,	\cite{liu2018runtime} proposes a lightweight machine learning based performance model to choose the number of threads to use for parallelization for a specific data size and operation. With the final goal of improving the training time in a neural network, \cite{liu2018runtime} selects 4 performance features collected by hardware counters namely, number of CPU cycles, number of cache misses, cache accesses for the last cache level, and number of level 1 cache hits. Then they take two different approaches to build their model. In the first on they try 10 different regression models including random forest, and in the second one they use hill climbing algorithm to choose the number of threads. In addition to hardware independent, and not requiring the training process, hill climbing algorithm achieves a much higher accuracy compared to the best performing regression model.

In this paper, we suggest using machine learning to directly predict the optimal chunk size to achieve the best performance instead of predicting the execution time or the optimal number of cores to run the application on. For this purpose, we have offered a set of general features that are not specific to an application and could easily be extracted at compile time or at run time. Once the data has been collected and our model has been created, the prediction results could be easily applied to a new application with a negligible overhead. 

%\cite{sun2017automated}
As another field to use machine learning, \cite{qawasmeh2015adaptive} collects seven runtime events and uses machine learning not to predict the performance, but to schedule the tasks. These events include, task creation, suspension, execution, completion, implicit/explicit barrier, parallel region, and finally loop/master/single region runtime events, collected through the OMPT using ORA API. Experimenting with four different machine learning techniques, including support vector machine, random forest, neural networks, and naive bayes, they would select one specific task pool configuration out of the three pre-defined options as the final classification result. Testing this framework on a real life molecular dynamics application, they observed an up to 31\% improvement in performance. 
%Compiler-based methods:

The authors of \cite{wang2009mapping} propose using machine learning to predict the optimal number of threads, and also the optimal scheduling policy for running an OpenMP application. Through that, they were able to develop an automatic compiler-based method to map a parallel application to a multicore processor. They collect three type of features namely, code, data, and runtime features. Code features are extracted from the code directly, and they include cycles per instruction, number of branches, load and store instructions, and computations per instruction. While the code features could be collected statically at compile time, the data and run-time features are collected through low-cost profiling runs. This group of features include loop iteration count, branch miss rate, and $L1$ data cache miss rate. The authors of \cite{wang2009mapping} then use an artificial neural network to predict the speedup achieved for a program with certain number of threads, and at the same time they use a support vector machine model to predict the best scheduling policy, out of block, cyclic, dynamic, and guided scheduling policies, for an unseen program.



	
	
	
	
%\cite{treibig2012performance} 
%profiling information about the application on a given architecture
%\cite{cammarota2012just},\cite{zhang2005runtime},\cite{thoman2012automatic}
%
%Machine learning models 
%\cite{singh2009real}, \cite{zomaya2001observations}, \cite{qawasmeh2015adaptive}
%
%
%
%	
%
%	
%
%\cite{li2009machine}
\vspace{\baselineskip}



\pagebreak
\singlespacing

\chapter{Method}\label{Method}
\doublespacing
%In this section we discuss how and for what purpose the HPX backend for Blaze was modified. Our work here heavily depends on the data.
\vspace{\baselineskip}	
\section{Parallelization in Blaze}
Depending on the operation and the size of operands, this assignment could be parallelized through four different backends, namely, HPX, OpenMP\cite{dagum1998openmp}, C++ threads, and Boost\cite{Boost}. 
Table~\ref{table2} shows the default value for some of the threshold for parallelization applied to operations performed in Blaze. It should be noted that these thresholds should be tuned based on the parallelization backend and also the system architecture.
%For matrix matrix multiplication alongside the mentioned thresholds, there exists another set of thresholds to switch between using Blaze kernels or BLAS kernels. 
%As stated in Section~\ref{Background} Blaze being based on smart expression templates, offers the option to use either Blaze or a highly optimized kernel for computations. This option is set at compile time through $BLAS_mode$ macro. If $BLAS_mode=1$, 

\vspace{\baselineskip}	
\begin{table}[H]
	\centering
	\resizebox{\textwidth}{!}
	{\begin{tabular}{|c | c |} 
			\hline
			Benchmark & Array size \\ [0.5ex] 
			\hline
			\hline
			$DVECDVCEADD$, $DVECDVECMULT$ & 38000\\ 	
			\hline
			$DMATDMATADD$ & 36100 elements equivalent to a $175\times{175}$ matrix \\
			\hline	
			$DMATDMATMULT$ & 3025 elements equivalent to a $55\times{55}$ matrix  \\
			\hline			
	\end{tabular}}
	
	\caption{List of some of the thresholds applied to the operations performed by Blaze, starting from which the operation is executed in parallel}
	\label{table2}
\end{table} 

\vspace{\baselineskip}	
\subsection{Implementation of HPX Backend}
As stated earlier, as an ET-based library, blaze performs the calculations when an expression is assigned to a target, which is implemented through the \textit{blaze::Assign} function.

The four mentioned backends, parallelize this assignment process through a parallel for-loop, in which at each iteration a specific section of each of the vectors or matrices(called a block) is selected and assigned to a core. Each core then performs the operation on the block they have been assigned to.  

Each backend uses their own method for parallelizing this for loop. For HPX backend, current implementation uses a HPX \textit{parallel::for\textunderscore loop} with static chunking policy and chunk size of 1. This way, knowing the number of cores to run the application on, we can divide the original matrix equally among the cores, while the order of assignment of blocks to the cores is known at compile time.  
Listings\ref{old_hpx_backend} shows the current implementation of the HPX backend in Blaze.

\begin{lstlisting}[basicstyle=\fontsize{8}{9}\selectfont,float,floatplacement=H,caption= {Previous implementation of Assign function for HPX backend in Blaze.}, label={old_hpx_backend}]
template< typename MT1   // Type of the left-hand side dense matrix
, bool SO1       // Storage order of the left-hand side dense matrix
, typename MT2   // Type of the right-hand side dense matrix
, bool SO2       // Storage order of the right-hand side dense matrix
, typename OP >  // Type of the assignment operation
void hpxAssign( DenseMatrix<MT1,SO1>& lhs, const DenseMatrix<MT2,SO2>& rhs, OP op )
{
using hpx::parallel::for_loop;
using hpx::parallel::execution::par;

BLAZE_FUNCTION_TRACE;

using ET1 = ElementType_t<MT1>;
using ET2 = ElementType_t<MT2>;

constexpr bool simdEnabled( MT1::simdEnabled && MT2::simdEnabled && IsSIMDCombinable_v<ET1,ET2> );
constexpr size_t SIMDSIZE( SIMDTrait< ElementType_t<MT1> >::size );

const bool lhsAligned( (~lhs).isAligned() );
const bool rhsAligned( (~rhs).isAligned() );

const size_t threads    ( getNumThreads() );
const ThreadMapping threadmap( createThreadMapping( threads, ~rhs ) );

const size_t addon1     ( ( ( (~rhs).rows() % threadmap.first ) != 0UL )? 1UL : 0UL );
const size_t equalShare1( (~rhs).rows() / threadmap.first + addon1 );
const size_t rest1      ( equalShare1 & ( SIMDSIZE - 1UL ) );
const size_t rowsPerThread( ( simdEnabled && rest1 )?( equalShare1 - rest1 + SIMDSIZE ):( equalShare1 ) );

const size_t addon2     ( ( ( (~rhs).columns() % threadmap.second ) != 0UL )? 1UL : 0UL );
const size_t equalShare2( (~rhs).columns() / threadmap.second + addon2 );
const size_t rest2      ( equalShare2 & ( SIMDSIZE - 1UL ) );
const size_t colsPerThread( ( simdEnabled && rest2 )?( equalShare2 - rest2 + SIMDSIZE ):( equalShare2 ) );

for_loop( par, size_t(0), threads, [&](int i)
{
const size_t row   ( ( i / threadmap.second ) * rowsPerThread );
const size_t column( ( i % threadmap.second ) * colsPerThread );

if( row >= (~rhs).rows() || column >= (~rhs).columns() )
return;

const size_t m( min( rowsPerThread, (~rhs).rows()    - row    ) );
const size_t n( min( colsPerThread, (~rhs).columns() - column ) );

if( simdEnabled && lhsAligned && rhsAligned ) {
auto       target( submatrix<aligned>( ~lhs, row, column, m, n ) );
const auto source( submatrix<aligned>( ~rhs, row, column, m, n ) );
op( target, source );
}
else if( simdEnabled && lhsAligned ) {
auto       target( submatrix<aligned>( ~lhs, row, column, m, n ) );
const auto source( submatrix<unaligned>( ~rhs, row, column, m, n ) );
op( target, source );
}
else if( simdEnabled && rhsAligned ) {
auto       target( submatrix<unaligned>( ~lhs, row, column, m, n ) );
const auto source( submatrix<aligned>( ~rhs, row, column, m, n ) );
op( target, source );
}
else {
auto       target( submatrix<unaligned>( ~lhs, row, column, m, n ) );
const auto source( submatrix<unaligned>( ~rhs, row, column, m, n ) );
op( target, source );
}
} );
}
\end{lstlisting}

What we suggest here is that, some prior knowledge for example, architecture of the system we are running the application on, the expression that has to be executed, number of cores of the system, size and type of the arrays we are dealing with, and etc. should be able to help us to achieve a higher performance.
For this purpose we introduced two parameters block\textunderscore{size} and chunk\textunderscore{size}.

\begin{lstlisting}[basicstyle=\fontsize{8}{9}\selectfont,float,floatplacement=H,caption= {New implementation of Assign function for HPX backend in Blaze.}, label={new_hpx_backend}]
template< typename MT1   // Type of the left-hand side dense matrix
, bool SO1       // Storage order of the left-hand side dense matrix
, typename MT2   // Type of the right-hand side dense matrix
, bool SO2       // Storage order of the right-hand side dense matrix
, typename OP >  // Type of the assignment operation
void hpxAssign( DenseMatrix<MT1,SO1>& lhs, const DenseMatrix<MT2,SO2>& rhs, OP op )
{
using hpx::parallel::for_loop;
using hpx::parallel::execution::par;

BLAZE_FUNCTION_TRACE;

using ET1 = ElementType_t<MT1>;
using ET2 = ElementType_t<MT2>;

constexpr bool simdEnabled( MT1::simdEnabled && MT2::simdEnabled && IsSIMDCombinable_v<ET1,ET2> );
constexpr size_t SIMDSIZE( SIMDTrait< ElementType_t<MT1> >::size );

const bool lhsAligned( (~lhs).isAligned() );
const bool rhsAligned( (~rhs).isAligned() );

const size_t threads    ( getNumThreads() );
const size_t numRows ( min( static_cast<std::size_t>( BLAZE_HPX_MATRIX_BLOCK_SIZE_ROW ), (~rhs).rows() ) );
const size_t numCols ( min( static_cast<std::size_t>( BLAZE_HPX_MATRIX_BLOCK_SIZE_COLUMN ), (~rhs).columns() ) );

const size_t rest1      ( numRows & ( SIMDSIZE - 1UL ) );
const size_t rowsPerIter( ( simdEnabled && rest1 )?( numRows - rest1 + SIMDSIZE ):( numRows ) );
const size_t addon1     ( ( ( (~rhs).rows() % rowsPerIter ) != 0UL )? 1UL : 0UL );
const size_t equalShare1( (~rhs).rows() / rowsPerIter + addon1 );

const size_t rest2      ( numCols & ( SIMDSIZE - 1UL ) );
const size_t colsPerIter( ( simdEnabled && rest2 )?( numCols - rest2 + SIMDSIZE ):( numCols ) );
const size_t addon2     ( ( ( (~rhs).columns() % colsPerIter ) != 0UL )? 1UL : 0UL );
const size_t equalShare2( (~rhs).columns() / colsPerIter + addon2 );

hpx::parallel::execution::dynamic_chunk_size chunkSize ( BLAZE_HPX_MATRIX_CHUNK_SIZE );

for_loop( par.with( chunkSize ), size_t(0), equalShare1 * equalShare2, [&](int i)
{
const size_t row   ( ( i / equalShare2 ) * rowsPerIter );
const size_t column( ( i % equalShare2 ) * colsPerIter );

if( row >= (~rhs).rows() || column >= (~rhs).columns() )
return;

const size_t m( min( rowsPerIter, (~rhs).rows()    - row    ) );
const size_t n( min( colsPerIter, (~rhs).columns() - column ) );

if( simdEnabled && lhsAligned && rhsAligned ) {
auto       target( submatrix<aligned>( ~lhs, row, column, m, n ) );
const auto source( submatrix<aligned>( ~rhs, row, column, m, n ) );
op( target, source );
}
else if( simdEnabled && lhsAligned ) {
auto       target( submatrix<aligned>( ~lhs, row, column, m, n ) );
const auto source( submatrix<unaligned>( ~rhs, row, column, m, n ) );
op( target, source );
}
else if( simdEnabled && rhsAligned ) {
auto       target( submatrix<unaligned>( ~lhs, row, column, m, n ) );
const auto source( submatrix<aligned>( ~rhs, row, column, m, n ) );
op( target, source );
}
else {
auto       target( submatrix<unaligned>( ~lhs, row, column, m, n ) );
const auto source( submatrix<unaligned>( ~rhs, row, column, m, n ) );
op( target, source );
}
} );
}
\end{lstlisting}
\vspace{\baselineskip}	
%\subsection{HPX \textit{for\textunderscore loop}}
%HPX \textit{for\textunderscore loop} takes an execution policy as first argument, which is set to \textit{dynamic\textunderscore chunk\textunderscore size} execution policy in case of HPX backend for Blaze.

\vspace{\baselineskip}	
\section{Experiments}
In order to capture the relationship between number of cores, \textit{chunk\textunderscore{size}}, \textit{block\textunderscore{size}}, and the performance, we ran a series of experiments with different of these parameters and measured the number of floating point operations per second performed. 

For these experiments ,at the first step we selected the $DMatDMatADD$ benchmark which was implemented in Blazemark. $DMatDMatADD$ benchmark is a level 3 BLAS function to perform matrix-matrix addition in the form of $A=B+C$, where $A$, $B$, $C$ are square matrices of the same size. For simplification we are only studying raw-major matrices at this point. Our final goal is to extend the work to cover arbitrary data layouts for arrays.

To avoid adding the scheduling overhead for small matrix sizes, Blaze uses a threshold to start parallelization, which is specific to the type of operation. For matrix-matrix addition, if the number of elements in the matrix is greater than 36100 elements(which is equivalent to a square matrix of size 190$\times$190) Blaze uses the configured backend to parallelize the assignment operation. For this reason, we start our experiments with matrix size of 200x200 and gradually increase the size to 1587$\times$1587. 
Table~\ref{table1} show the matrix sizes and the number of cores chosen for our experiments with $DMATDMATADD$ benchmark.

\vspace{\baselineskip}	
\begin{table}[H]
	\centering
		\resizebox{\textwidth}{!}
		{\begin{tabular}{|c | c |} 
			\hline
			Matrix sizes & 200, 230, 264, 300, 396, 455, 523, 600, 690, 793, 912, 1048, 1200, 1380, 1587 \\ [0.5ex] 
			\hline
			Number of cores & 1, 2, 3, 4, 5, 6, 7, 8 \\ 	
			\hline
			Number of rows in the block & 4, 8, 12, 16, 20, 32 \\
			\hline	
			Number of columns in the block & 64, 128, 256, 512, 1024 \\
			\hline
			Chunk size & Between 1 and total number of blocks (logarithmic increase)\\\hline
		\end{tabular}}

		\caption{List of different values used for each variable for running the $DMATDMATADD$ benchmark}
		\label{table1}
\end{table}

\vspace{\baselineskip}	
Figure~\ref{fig1} shows the results of running $DMatDMatADD$ benchmark for matrix sizes and number of cores listed in Tbale~\ref{table1} based on grain size. 

On the other hand, Figure~\ref{fig4} integrates the results obtained from running the same benchmark with different matrix sizes. Each color in this graph represents a specific matrix size. 

\vspace{\baselineskip}	
\begin{figure}[H]
	\centering
	\hspace*{-2cm}\includegraphics[scale=.75]{images/fig13.png}
	\caption{The results obtained from running $DMATDMATADD$ benchmark through Blazemark for matrix size 690$\times$690 on different number of cores.}	
	\label{fig9}
\end{figure}

\vspace{\baselineskip}	
\begin{figure}[H]
	\centering
	{\includegraphics[scale=0.7]{images/fig2.png}}
	{\includegraphics[scale=0.7]{images/fig3.png}}
	\caption{The results obtained from running $DMATDMATADD$ benchmark through Blazemark for matrix of size 690$\times$690 from two different angles}	
	\label{fig1}
\end{figure}

\begin{figure}[H]
	\centering
	\includegraphics[width=1\linewidth]{images/fig4.png}
	\caption{The results obtained from running $DMATDMATADD$ benchmark through Blazemark for matrix sizes from 200$\times$200 to 1587$\times$1587}	
	\label{fig4}
\end{figure}

\vspace{\baselineskip}	
\subsection{Observation}
The final purpose of our experiments is to find a chunk size that gives us the best performance for a given matrix size on a given machine. This chunk size should also be tailored to the expression being executed, and this all is based on assuming that we have already fixed the block size.
So the first step appeared to be selecting the block size. For this purpose, we ran the experiments with a selection of block sizes as shown in Table~\ref{table1}.


It should be mentioned that there were three constraints on selecting the block sizes. First, Blaze forces the number of columns in a raw-major matrix to be divisible to SIMD register size in order to be able to take advantage of vectorization. Second, we have selected the number of columns in our blocks to be either divisible by cache line or to contain all the columns of the matrix.     


\begin{figure}[H]
	\centering
	\includegraphics[width=1\linewidth]{images/fig5.png}
	\caption{The results obtained from running $DMATDMATADD$ benchmark through Blazemark for matrix sizes from 690$\times$690 with different combinations of block size and chunk size on $4$ cores}	
	\label{fig5}
\end{figure}

The collected data, as seen in Figure~\ref{fig5}, suggests two main points:
\begin{itemize}
	\item For each selected block size, there is a range of chunk sizes that gives us the best performance. 
	\item Except for some uncommon cases, no matter which block size we choose, we are able to achieve the maximum performance if we select the right chunk size.  
\end{itemize}

This motivated us to move our search parameter from chunk size to grain size. As stated earlier, grain size is the amount of work assigned to one HPX thread. Here we represent grain size by number of floating point operations performed by a HPX thread. For example, performing addition among two matrices, if we choose the block size as $4\times64$ and chunk size as $3$, the grain size would be $3\times4\times64=768$. 
Note that in our experiments whenever the number of columns of the original matrix is not divisible to the selected number of columns for block size, there would be a set of blocks with less number of elements than the selected block size, this has been considered when calculating the grain size.  

By changing our focus to the grain size instead of the block size and the chunk size, Figure~\ref{fig6} shows how the throughput changes with regards to the grain size for the $DMATDMATADD$ benchmark, for each specific block size. Each combination of block size and chunk size generates a point in the graph. On the other hand, Figure~\ref{fig9} looks at these graphs from another aspect, keeping the problem size constant but changing the number of the cores to run the benchmark on, instead.

\begin{figure}[H]
	\centering\includegraphics[width=1\linewidth]{images/fig6.png}
	\caption{The results obtained from running $DMATDMATADD$ benchmark through Blazemark for matrix size 690$\times$690 on $4$ cores.}	
	\label{fig6}
\end{figure}

%\begin{figure}[H]
%	\centering
%	\hspace*{-2cm}\includegraphics[scale=.75]{images/fig8.png}
%	\caption{The results obtained from running $DMATDMATADD$ benchmark through Blazemark for 5 different matrix sizes on $4$ cores.}	
%	\label{fig7}
%\end{figure}

\begin{figure}
	\subfloat[]
	{\centering\includegraphics[scale=.75]{images/fig11.png}	
	\label{fig8:a}}

	\subfloat[]{
	\centering\includegraphics[scale=.75]{images/fig12.png}
	\label{fig8:b}}
	\caption{Throughput vs. grain size graph obtained from running $DMATDMATADD$ benchmark  on $4$ cores for matrix sizes (a) smaller than 793$\times$793 and (b) larger than 793$\times$793.}
	\label{fig8}	
\end{figure}



\vspace{\baselineskip}	
\section{Method}
Looking at the throughput vs. grain size graphs and the consistent pattern observable motivated us to try to model the relationship between throughput and grain size. 
In order to simplify the process and eliminate the effect of different possible factors, we started with limiting the problem to a fixed matrix size. 

\vspace{\baselineskip}	
\subsection{Polynomial Fit}
In our first attempt we used a 2nd degree polynomial to model throughput against grain size. For each matrix size, we fitted the corresponding graphs shown in Figure~\ref{fig8} to a second degree polynomial. 

\begin{figure}[H]
	\centering
	\subfloat[]{\includegraphics[scale=.45]{images/polyfit/fig_1_690.png}\label{fig10:a}}
	\subfloat[]{\includegraphics[scale=.45]{images/polyfit/fig_2_690.png}\label{fig10:b}}{\hfill}
	\subfloat[]{\includegraphics[scale=.45]{images/polyfit/fig_3_690.png}\label{fig10:c}}
	\subfloat[]{\includegraphics[scale=.45]{images/polyfit/fig_4_690.png}\label{fig10:d}}{\hfill}
	\subfloat[]{\includegraphics[scale=.45]{images/polyfit/fig_5_690.png}\label{fig10:e}}
	\subfloat[]{\includegraphics[scale=.45]{images/polyfit/fig_6_690.png}\label{fig10:f}}{\hfill}
	\subfloat[]{\includegraphics[scale=.45]{images/polyfit/fig_7_690.png}\label{fig10:g}}
	\subfloat[]{\includegraphics[scale=.45]{images/polyfit/fig_8_690.png}\label{fig10:h}}
							
	\caption{The results of fitting the throughput vs grain size data into a 2d polynomial for $DMATDMATADD$ benchmark for matrix size 690$\times$690 with different number of cores on the test data set (a) 1 core, (b) 2 cores, (c) 3 cores, (d) 4 cores, (e) 5 cores, (f) 6 cores, (g) 7 cores, (h) 8 cores.}	
	\label{fig10}
\end{figure}


\vspace{\baselineskip}	
Figure~\ref{fig10} shows the results of using a quadratic function to fit the data for one matrix size with different number of threads. We used th \textit{polyfit} package from \textit{numpy} library in \textit{python} for this purpose, which tries to minimize the least-square error over all the samples.  

For our experiment, we divided the data into two sections, training and test. $60\%$ of the data was randomly chosen for the training part and the rest was considered as the test set. The training set was used to find the best 2nd degree polynomial for the data, and once the parameters were identified, the generated 2nd degree polynomial was applied to the test set to measure how good our fit was performing. 

For the matrix size $690\times690$ our dataset contained 117 data points, 72 of which was randomly selected to build the model. The mean relative error for each number of cores, calculated using Equation~\ref{eq2}, is represented in Figure~\ref{fig11} for training and test set. In this equation, $t_i$ and $p_i$ denote the true value and the predicted value of the $i$th sample respectively, where $n$ is the number of samples with the particular number of cores. 

\begin{equation}{\label{eq2}}
Relative\,\,Error = \frac{1}{n}\sum_{i=1}^{n} {1-p_i/t_i}
\end{equation}

\vspace{\baselineskip}	
\begin{figure}[H]
	\centering
	\includegraphics[scale=.75]{images/polyfit/fig_train_test_690.png}
	\caption{The training and test error for fitting data obtained from the $DMATDMATADD$ benchmark for matrix size $690\times690$ against different number of cores cores.}	
	\label{fig11}
\end{figure}

\vspace{\baselineskip}	
\subsubsection{Generalizing the fitted function to include number of cores}
In this step, we try to generalize the fitted 2nd degree polynomial obtained from the previous step, represented by $P=ag^2+bg+c$, where $P$ is the throughput and $g$ is the grain size,by looking at how the three parameters $a$, $b$, and $c$ change when number of cores changes. 
A $3$rd degree polynomial seems to a reasonable fit for each of these parameters, in regards to number of cores. In order to avoid overfitting, we excluded two of the data points($2$ and $5$) from the data points used for fitting the polynomial and tested the fitted function on those two points to see how well the function is working on unseen data points. 


\vspace{\baselineskip}	
\begin{figure}[H]
	\centering
	\subfloat[]{\includegraphics[scale=.3]{images/polyfit/fig_690_params_0.png}\label{fig15:a}}
	\subfloat[]{\includegraphics[scale=.3]{images/polyfit/fig_690_params_1.png}\label{fig15:b}}
	\subfloat[]{\includegraphics[scale=.3]{images/polyfit/fig_690_params_2.png}\label{fig15:c}}
	\caption{Fitting the parameters of the quadratic function with a $3$rd degree polynomial from the $DMATDMATADD$ benchmark for matrix size $690\times690$ against different number of cores.}	
	\label{fig15}
\end{figure}

\vspace{\baselineskip}	
\begin{figure}[H]
	\centering
	\includegraphics[scale=.45]{images/polyfit/fig_690_params_error.png}
	
	\caption{The error in fitting the parameters $a$, $b$, and $c$ for matrix size $690\times690$.}	
	\label{fig16}
\end{figure}

Using this $3$rd degree polynomial to fit the parameters, we can generalize the relationship between throughput and grain size in the following equation:

\begin{equation}\label{eq3}
P=a_{11}g^2N^3+a_{10}g^2N^2+...+a_1N+a_0
\end{equation}
where $P$ is the throughput, $g$ is the grain size, and $N$ is the number of cores and coefficients $a_{11},...,a_0$ are the real values.

Knowing that a polynomial of degree $2$ in terms of grain size and of degree $3$ in terms of number of cores, we can try to fit our original data directly to the above mentioned formula (Equation~\ref{eq3}). The results of the original data obtained from running $DMATDMATADD$ benchmark, the fitted polynomial based on Equation~\ref{eq3}, is represented in Figure~\ref{fig18}, for 2,4, and 8 cores for a matrix of size $690\times690$.

\vspace{\baselineskip}	
\begin{figure}[H]
	\centering
	\subfloat[]{\includegraphics[scale=.35]{images/polyfit/fig_690_total_2.png}\label{fig18:a}}
	\subfloat[]{\includegraphics[scale=.35]{images/polyfit/fig_690_total_4.png}\label{fig18:b}}
	\subfloat[]{\includegraphics[scale=.35]{images/polyfit/fig_690_total_8.png}\label{fig18:c}}
	\caption{Results of fitting the data from $DMATDMATADD$ benchmark with a polynomial of degree $2$ in terms of grain size and of degree $3$ in terms of number of cores for matrix size $690\times690$ for (a) 2 core, (b) 4 cores, (c) 8 cores.}	
	\label{fig18}
\end{figure}

\vspace{\baselineskip}	
\begin{figure}[H]
	\centering
	\subfloat[]{\includegraphics[scale=.45]{images/polyfit/fig_690_total_error.png}\label{fig17:a}}
	\subfloat[]{\includegraphics[scale=.45]{images/polyfit/fig_690_total_error_corrected.png}\label{fig17:b}}
	\caption{The training and test error obtained fitting the data to a polynomial of degree $2$ in terms of grain size and of degree $3$ in terms of number of cores for matrix size $690\times690$, for each number of cores. (a) All the data points are include in caluculation of error, (b) the leftmost sample was removed from error calculation.}	
	\label{fig17}
\end{figure}

Figure~\ref{fig17:a} shows the obtained relative error on both training and test sets. The graph suggests a higher test error compared to the training error, mostly caused by the left hand side of the graph. The effect of removing the leftmost sample from error calculations is depicted in Figure~\ref{fig17:b}.

Although we are interested in finding a model that results in a low training and test error, our purpose is mainly finding the region that generates the highest performance. So, even though our model might not match the original data in all data points, due to having a different nature than a quadratic function, our focus would be on how this fit can help us to find which range of grain sizes, or how big the task sizes should be, to achieve the highest performance. 



\vspace{\baselineskip}	
\subsubsection{Finding the range of grain size to achieve the highest performance}
The major advantage of using a quadratic function to fit the data in terms of grain size, when number of cores is fixed, is the simplicity of the formula, which makes it possible for us to find the peak of the graph very easily. n order to add some uncertainty to our prediction, instead of finding the maximum of the quadratic function, we identified the range of grain size that results in a performance within $10\%$ of the maximum performance. For a second degree polynomial in terms of $g$, $P=ag^2+bg+c$, the minimum or maximum of the polynomial is located at $p^{*}=\frac{-b}{2a}$, and $a$, $b$, $c$ are 3rd degree polynomials of number of cores.    

\vspace{\baselineskip}	
\begin{figure}[H]
	\centering
	\subfloat[]{\includegraphics[scale=.3]{images/polyfit/fig_690_total_2_range.png}\label{fig12:a}}
	\subfloat[]{\includegraphics[scale=.3]{images/polyfit/fig_690_total_4_range.png}\label{fig12:b}}
	\subfloat[]{\includegraphics[scale=.3]{images/polyfit/fig_690_total_8_range.png}\label{fig12:c}}
	\caption{The range of grain size (shown as the red line) that leads to a performance within $10\%$ of the maximum performance for (a) 2 cores, (b) 4 cores and (b) 8 cores.}	
	\label{fig12}
\end{figure}


\begin{figure}[H]
	\centering
	\subfloat[]{\hfill\includegraphics[scale=.5]{images/polyfit/fig_690_peak_range_all.png}\label{fig13:a}}
	\subfloat[]{\includegraphics[scale=.5]{images/polyfit/fig_523-912_peak_range_all.png}\label{fig13:b}}
	\caption{The range of grain size within $10\%$ of the maximum performance of the fitted polynomial function for $DMATDMATADD$ benchmark for different number of cores for (a) matrix size $690\times690$ (b)matrix size $523\times523$ to $912\times912$.}	
	\label{fig13}
\end{figure}



Figure~\ref{fig13:a} shows the calculated range for matrix size $690\times690$ for each specific number of threads, while Figure~\ref{fig13:b} compares the range for different matrix sizes. 
\vspace{\baselineskip}	

\subsubsection{Estimating the chunk size}
Once we identified a range of grain sizes that is expected to leads us to highest achievable performances for a specific matrix size and a specific number of cores, the next step is finding the possible combinations of block size and chunk size to achieve that range of grain sizes.  
As stated earlier in this chapter, results obtained from Figure~\ref{fig6} suggests that with a fixed grain size, our choice of block size does not affect the performance directly, as long as there exist a chunk size that when combined by the block size could result in the specified grain size. 

In our experiment, we selected our block size to be $4\times256$. With this assumption, in order for the grain size to be within the specified range for each matrix size, chunk size has to be within a specific range size too.

For example, for a $690\times690$ matrix we calculated the range of maximum performance for $4$ cores to be $[3.88, 4.92]$ in logarithmic scale which is equivalent to $[7586, 83176]$. Setting the block size to $4\times256$, this range forces the chunk size to be within the range $[9,90]$. The range of chunk sizes to match the range of grain sizes identified, and their corresponding throughput is shown in Figure~\ref{fig14}, for matrix size $690\times690$ and block size $4\times256$. The green line is the throughput achieved by the current implementation of HPX backend. Since the graph from the original data is skewed to right, we selected the point after the median of all the chunk sizes in the range, as our candidate chunk size for this specific configuration.


  
\vspace{\baselineskip}	
\begin{figure}[H]
	\centering
	\subfloat[]{\includegraphics[scale=.35]{images/polyfit/fig_690_chunks_2_4-256.png}\label{fig14:a}}
	\subfloat[]{\includegraphics[scale=.35]{images/polyfit/fig_690_chunks_4_4-256.png}\label{fig14:b}}
	\subfloat[]{\includegraphics[scale=.35]{images/polyfit/fig_690_chunks_8_4-256.png}\label{fig14:c}}
	\hfill
	\subfloat[]{\includegraphics[scale=.35]{images/polyfit/fig_690_chunks_2_4-512.png}\label{fig14:d}}
	\subfloat[]{\includegraphics[scale=.35]{images/polyfit/fig_690_chunks_4_4-512.png}\label{fig14:e}}
	\subfloat[]{\includegraphics[scale=.35]{images/polyfit/fig_690_chunks_8_4-512.png}\label{fig14:f}}
	\caption{The range of chunk sizes to produce a grain size within $10\%$ of the maximum performance of the fitted quadratic function for $DMATDMATADD$ benchmark for matrix size $690\times690$ with block size of $4\times256$ on (a) $2$ cores, (b) $4$ cores, and (c) $8$ cores, and block size of $4\times512$ on (d) $2$ cores, (e) $4$ cores, and (f) $8$ cores. Silver points denotes the detected range of chunk size, and the red star shows the median point.}	
	\label{fig14}
\end{figure}


\subsection{Bathtub model}
In the previous section we studied the possibility of using a polynomial to capture the relationship between grain size, number of cores, and throughput for a fixed matrix size, with the purpose of finding a range of grain size that leads us to maximum performance. 
Although the polynomial function was helpful in directing us toward our objective, it does not have a physical implication. 

This motivated us to change our view, and instead of looking just at the data and trying to find a function to fit the data, study the behavior of the data, and then find a function that would be likely to fit the data. That function would be a good fit mostly because that's how we expect the throughput to change with grain size, and not just how the data looks like.   

As stated in Section~\ref{task}, Grubel et.al.\cite{grubel2015performance} has studied the task granularity for a specific problem(1D stencil). Looking at the graph representing the execution time based on the grain size, which resembles a bathtub, we are interested in formulating this graph based on our understanding of the effect of task granularity. Figure~\ref{fig21} shows the execution time in terms of grain size for $DMATDMATADD$ benchmark for matrix size $690\times690$ on $4$ cores.

\vspace{\baselineskip}	
\begin{figure}[H]
	\centering
	\subfloat[]{\includegraphics[scale=.5]{images/bathtub/all_690_4.png}\label{fig20:a}}
	\subfloat[]{\includegraphics[scale=.5]{images/bathtub/tasks_all_690_4.png}\label{fig20:b}}
	\caption{(a)The execution time vs. grain size graph, and (b) execution time vs. number fo tasks graph for $DMATDMATADD$ benchmark for matrix size $690\times690$ ran on $4$ cores.}	
	\label{fig21}
\end{figure}

\vspace{\baselineskip}	
For the sake of simplicity, we change the x axis from grain size to number of tasks. Each specific grain size would create a specific number of tasks(HPX threads), since the parameters we are interested in are directly associated with he number of tasks, we represent execution time based on number of tasks, as shown in Figure~\ref{fig20:b}.
 

Looking at the left hand side of the graph in Figure~\ref{fig20:b}, we can observe that for the first three points, the number of tasks created is smaller than the number of cores(which is 4 in this example). This means that in any of these cases there is at least one idle core, while the other cores are assigned a rather big chunk of work. The performance degradation we observe in that points is associated with starvation, meaning that we are not utilizing our computation resources to the full extent. In these three points, the number of cores actually doing the work is equal to the number of the tasks, since each core gets to execute at most one task.    

To generalize the problem, assuming we are running our application on $N$ cores, with a grain size equal to $g$, $n_t$ tasks are being created, and $M$ cores are actually doing the work. If $n_t<N$, $M$ would be equal to $n_t$, otherwise $M=N$.


From overhead point of view though, if we represent the overhead of creating one task on a particular machine as $\alpha$, the overhead of creating $d$ tasks would be $n_t\alpha$, but this overhead is divided between the $M$ cores actually doing the work. 

To summarize, knowing the grain size, we are expecting the execution time in a many-task runtime system to be mainly affected by these factors, the overhead of creating one task($\alpha$), the number of cores that are actually doing the work($M$), the sequential execution time($t_s$), and finally the portion of the program that could actually be parallelized($\gamma$). 

If we try to integrate these information into a formula, we would expect the relation between execution time($t$) and number of tasks($n_t$) as follows:
\begin{equation}\label{new}
t=\frac{\alpha{n_t}+t_s}{M}+\gamma
\end{equation}

which could be decomposed into these two equations:

\begin{equation}
t=\left\{
\begin{aligned}
\alpha+\frac{t_s}{n_t}+\gamma  \:\:\:\:\:\:\:\:      \text{ if } n_t<N\\
\frac{\alpha{n_t}+t_s}{N}+\gamma\:\:\:\:\:\:\:\:     \text{otherwise}
\end{aligned}
\right.
\end{equation}

Now we use this function to find the best three parameters $\alpha$, $t_s$, and $\gamma$ so that the collected data would fit this model. For this purpose we used the \textit{curve\textunderscore{fit}} package from \textit{SciPy} library in \textit{python}.

In order to make Equation~\ref{new} differentiable, we used the softplus function(Equation~\ref{softplus}) to represent $M$ based on $n_t$.
\begin{equation}\label{softplus}
f(x)=Ln(1+e^x)
\end{equation}

Which results in Equation~\ref{soft_new}:
\begin{equation}\label{soft_new}
t=\frac{\alpha{n_t}+t_s}{(N-1)-Ln(1+(e^{N-1}-1)e^{-n_t})}+\gamma
\end{equation}

Here again we limited our problem to one specific matrix size at a time, and divided the whole data for each matrix size and number of cores into two parts, $60\%$ for training and $40\%$ for testing.


\vspace{\baselineskip}	
\begin{figure}[H]
	\centering
	\subfloat[]{\includegraphics[scale=.45]{images/bathtub/pred/pred_690_4.png}\label{fig22:a}}
	\subfloat[]{\includegraphics[scale=.45]{images/bathtub/pred/pred_690_8.png}\label{fig22:b}}	
	\caption{(a)The execution time vs. grain size graph, and (b) execution time vs. number fo tasks graph for $DMATDMATADD$ benchmark for matrix size $690\times690$ ran on $4$ cores.}	
	\label{fig22}
\end{figure}
\vspace{\baselineskip}	
Figure~\ref{fig23} represents the relative error calculated for both training and test set, which is less than $5\%$.

\vspace{\baselineskip}	
\begin{figure}[H]
	\centering
	{\includegraphics[scale=.45]{images/bathtub/error_690.png}}	
	\caption{The error in fitting execution time with the bathtub formula for $DMATDMATADD$ benchmark for matrix size $690\times690$ with different number of cores.}	
	\label{fig23}
\end{figure}

\vspace{\baselineskip}	
Assuming that this function fits our data in acceptable manner, next step would be to check how these three parameters change with the number of cores.

\begin{equation}\label{usl_fit}
f(x)=\frac{m_0}{x}+\frac{m_1(x-1)}{x}+m_2(x-1)+m_3(x)(x-1)
\end{equation}


\vspace{\baselineskip}	
\begin{figure}[H]
	\centering
	\subfloat[]{\includegraphics[scale=.3]{images/bathtub/coef_1_690.png}\label{fig24:a}}
	\subfloat[]{\includegraphics[scale=.3]{images/bathtub/coef_2_690.png}\label{fig24:b}}	
	\subfloat[]{\includegraphics[scale=.3]{images/bathtub/coef_3_690.png}\label{fig24:c}}
	\caption{Fitting the three parameters (a)$\alpha$, (b)$t_s$, and (c)$\gamma$ for $DMATDMATADD$ benchmark for matrix size $690\times690$.}	
	\label{fig24}
\end{figure}


\vspace{\baselineskip}	
We can integrate Equation~\ref{new} and Equation~\ref{usl_fit} to predict the execution time for a given matrix size and number of cores.
For each matrix size having found the parameters $m_0$ to $m_3$, we can find $\alpha$, $t_s$, and $\gamma$ for the particular number of cores requested through Equation~\ref{usl_fit}. Then we can plug in the calculated values for $\alpha$, $t_s$, and $\gamma$ into Equation~\ref{new} to predict the execution time.

Figure~\ref{fig25} shows the prediction error on the test set for $DMATDMATADD$ benchmark for matrix size $690\times690$. The axis shows the different samples in the test set, and the label of each point represents the number of tasks created for that particular data point. As it could be seen, certain number of tasks result in higher prediction error, which needs to be studied.


\vspace{\baselineskip}	
\begin{figure}[H]
	\centering	{\hfill\includegraphics[scale=.35]{images/bathtub/prediction_error_overall690.png}}	
	\caption{The error in fitting execution time with the bathtub formula for $DMATDMATADD$ benchmark for matrix size $690\times690$ with different number of cores.}	
	\label{fig25}
\end{figure}



\vspace{\baselineskip}	 
\pagebreak
\singlespacing

\chapter{Understanding the effect of grain size on concurrency in an asynchronous many-task runtime system}\label{Bathtub}
\doublespacing
\subsection{Blazemark}
Blazemark is a benchmark suite provided by Blaze to compare the performance of Blaze with other linear algebra libraries including Blitz++\cite{Blitz}, Boost uBLAS\cite{Boost}, GMM++\cite{GMM++}, Armadillo\cite{sanderson2016armadillo}, MTL4\cite{MTL}, and Eigen3\cite{Eigen}, alongside plain BLAS libraries like Atlas\cite{ATLAS}, Goto\cite{gotoblas}, and Intel MKL.\cite{MKL}

\begin{figure}[H]
	\centering
	\includegraphics[scale=0.5]{images/blazemark_1.png}
	\includegraphics[scale=0.5]{images/blazemark_2.png}
	\caption{An example of the results obtained from running $DVECDVECADD$ benchmark through Blazemark}	
	\label{blazemark1}
\end{figure}


\subsection{Setup}
Our experiments were run on Marvin nodes of Rostam cluster at Center for Computation and Technology(CCT) at Louisiana State University. Table~\ref{table3} and Table~\ref{4} show some of the specifications of this node.

\vspace{\baselineskip}	
\begin{table}[H]
	\centering
%	\resizebox{\textwidth}{!}
	\scalebox{0.75}
	{\begin{tabular}{|c | c |} 
			\hline
			CPU &  2 x Intel(R) Xeon(R) CPU E5-2450 0 @ 2.10GHz \\ [0.5ex] 
			\hline
			RAM & 48 GB\\ 	
			\hline
			Number of Cores & 16\\
			\hline	
			Hyperthreading & Off \\
			\hline			
	\end{tabular}}	
	\caption{Specifications of the Marvin node from Rostam cluster at CCT.}
	\label{table3}
\end{table} 


\vspace{\baselineskip}	
\begin{table}[H]
	\centering
	\scalebox{0.9}
	{\begin{tabular}{|c | c | c | c | c|} 
			\hline
			Cache Level &  Coherency Line Size & Number of Sets & Ways of Associativity & Size\\ [0.5ex] 
			\hline
			1 & 64 & 512 & 8 & 32KB \\	
			\hline
			2 & 64 & 512 & 8 & 256KB \\
			\hline	
			3 & 64 & 512 & 20 & 20480KB \\
			\hline			
	\end{tabular}}	
	\caption{Cache specifications of the Marvin node from Rostam cluster at CCT.}
	\label{table4}
\end{table} 
\vspace{\baselineskip}	

\vspace{\baselineskip}	
\begin{table}[H]
	\centering
	%	\resizebox{\textwidth}{!}
	\scalebox{0.75}
	{\begin{tabular}{|c | c |} 
			\hline
			HPX version commit & 1.3.0 \\ [0.5ex] 
			\hline
			Blaze version & 3.5\\ 	
			\hline
			
	\end{tabular}}	
	\caption{Specifications of the libraries used to run our experiments.}
	\label{table5}
\end{table}
%Marvin:
%cache level 1
%coherency line size: 64
%number of sets: 512
%ways of associativity: 8
%type: Instruction
%size: 32K
%
%cache level 2
%coherency line size: 64
%number of sets: 512
%ways of associativity: 8
%type: Unified
%size: 256K
%
%cache level 3
%coherency line size: 64
%number of sets: 512
%ways of associativity: 20
%type: Unified
%size: 20480K
%
%
%Trillian:
%cache level 1
%coherency line size: 64
%number of sets: 64
%ways of associativity: 4
%type: Data
%size: 16K
%
%cache level 1
%coherency line size: 64
%number of sets: 512
%ways of associativity: 2
%type: Instruction
%size: 64K
%
%cache level 2
%coherency line size: 64
%number of sets: 2048
%ways of associativity: 16
%type: Unified
%size: 2048K

%cache level 3
%coherency line size: 64
%number of sets: 2048
%ways of associativity: 48
%type: Unified
%size: 6144K



\pagebreak
\singlespacing

\chapter{Splittable Task}\label{splittable_task}
\doublespacing
\section{splittable Task}

In the previous chapter we proposed an analytical model to estimate the execution time of a balanced parallel for loop in terms of the grain size. Based on this model, we offered an approach to find the range of grain size to achieve minimum execution time. The parameters of the proposed model are identified through a benchmark and are exposed to the Blaze library to predict the range of grain size for minimum execution time of a problem at run-time. So the proposed method is a combination of compile-time and run-time solution to improve the performance.\\
In this chapter we choose another direction and look into a run-time adaptive solution to control task granularity in order to achieve the minimum execution time. Why? unbalanced work load.

Utilizing splittable tasks is a runtime adaptive method for managing task granularity, to avoid the large overhead of creating and managing too many tasks due to the fine grain parallelism on one hand, and the starvation resulted from creating less tasks than the available parallelism on the other hand.\\ 
Splittable tasks are tasks that could be partitioned into smaller tasks, when sufficient parallelism is available~\cite{prell2016embracing}. 

~\cite{robison2008optimization}

Prell intensively studies using the splittable tasks for runtime adaptivity in ~\cite{prell2016embracing}. They start by offering steal-half strategy for work stealing in order to steal half of the tasks from a worker's queue at each steal attempt instead of just one task. This could help to avoid creating too much work stealing overhead. But depending on the application, steal-one or steal-half might be the preferable method for work stealing. They propose an adaptive method to decide on whether to use steal-one or steal-half strategy at runtime, based on the current selected strategy and the ratio of the number of executed tasks($M$) within the last $N$ steals, where $N$ is considered the evaluation interval~\cite{prell2016embracing}.\\ 
Next, basing on lazy task scheduling~\cite{tzannes2014lazy}, they suggest to instead of creating at the tasks and decide on how to the workers should steal them, we can create one task and let it split if needed. This way you wouldn't have to deal with the overhead of scheduling tasks along with the overhead of stealing the tasks.\\
At each split, a portion of the task would be added to the current worker's deque as a splittble task to be stolen by free workers, while the rest of the task would be executed by the current worker.    
They propose two splitting strategies namely, and adaptive. 


Our work here is based on Prell~\cite{prell2016embracing}'s definition of splittable tasks and their split strategies. We have implemented an executor within HPX which would create splittable tasks to execute the work, and we improved upon their work in following ways:

\begin{itemize}
\item At each split, instead of allowing the created splittable task to be stolen by free workers, we turn work stealing off, identify the idle workers and explicitly assign the work to one of them. This way we avoid the work stealing overhead originated from unsuccessful steal attempts. 
\item We suggest to utilize our proposed method for identifying the range of grain size for minimum execution time, and use the lower-bound of the range as a cutoff value to stop splitting in order to avoid creating too fine tasks.   
\item We used Apex, an Autonomic Performance Environment for Exascale~\cite{huck2015autonomic}, as a profiling tool to study how tasks are being scheduled.
\end{itemize}




\subsection{Implementation}
For a parallel for-loop with the range of $[a,b)$, one splittable task containing all the iterations from $a$ to $(b-1)$ would be created. Depending on the splitting strategy when a certain condition is met this task would be split into two tasks, $task_1$ containing iterations in the range of $[a,c)$ and $task_2$ would contain iterations from $[c,b)$, where $c$ is the split point. $task_1$ would be executed by the current worker while $task_2$, which is also a splittable task, would be scheduled to run on another core. This allows the runtime to adaptively decide whether to split the current task into smaller tasks or just run it.     

 
\subsection{Splitting Strategies}
The splittable task could be split either based on total number of cores available, or number of idle cores at the time of split. 

\subsubsection{Splitting based on total number of cores}

\subsubsection{Splitting based on number of idle cores}

\subsection{Results}

\subsubsection{For-loop Benchmark}

\begin{figure}[H]
	\centering
	\subfloat[]
	{\centering\includegraphics[scale=.35]{images/hpx_for_loop/splittable/all_idle_cores/marvin_10000_1.png}	
		\label{fig66:a}}
	\subfloat[]
	{\centering\includegraphics[scale=.35]{images/hpx_for_loop/splittable/all_idle_cores/marvin_10000_2.png}	
		\label{fig66:b}}\hfill
	\subfloat[]
	{\centering\includegraphics[scale=.35]{images/hpx_for_loop/splittable/all_idle_cores/marvin_10000_3.png}	
		\label{fig66:c}}
	\subfloat[]
	{\centering\includegraphics[scale=.35]{images/hpx_for_loop/splittable/all_idle_cores/marvin_10000_4.png}	
		\label{fig66:d}}\hfill
	\subfloat[]
	{\centering\includegraphics[scale=.35]{images/hpx_for_loop/splittable/all_idle_cores/marvin_10000_5.png}	
		\label{fig66:e}}
	\subfloat[]
	{\centering\includegraphics[scale=.35]{images/hpx_for_loop/splittable/all_idle_cores/marvin_10000_6.png}	
		\label{fig66:f}}\hfill
	\subfloat[]
	{\centering\includegraphics[scale=.35]{images/hpx_for_loop/splittable/all_idle_cores/marvin_10000_7.png}	
		\label{fig66:g}}
	\subfloat[]
	{\centering\includegraphics[scale=.35]{images/hpx_for_loop/splittable/all_idle_cores/marvin_10000_8.png}	
		\label{fig66:h}}\hfill
	\caption{The results of running the hpx for loop using splittable tasks with all-cores and idle-cores split types compared with different grain sizes, for $problem\_size=10000$, for (a) 1 core, (b) 2 cores, (c) 3 cores, (d) 4 cores, (e) 5 cores, (f) 6 cores, (g) 7 cores, (h) 8 cores. The unit for execution time is microseconds.}
	\label{fig66}	
\end{figure}

\subsubsection{Blazemark}

\subsubsection{Apex}
\pagebreak
\singlespacing

\chapter{Setup}\label{Results}
\doublespacing
\section{Bathtub model}
In the previous section we studied the possibility of using a polynomial to capture the relationship between grain size, number of cores, and throughput for a fixed matrix size, with the purpose of finding a range of grain size that leads us to maximum performance. 
Although the polynomial function was helpful in directing us toward our objective, it does not have a physical implication. 

This motivated us to change our view, and instead of looking just at the data and trying to find a function to fit the data, study the behavior of the data, and then find a function that would be likely to fit the data. That function would be a good fit mostly because that's how we expect the throughput to change with grain size, and not just how the data looks like.   

As stated in Section~\ref{task}, Grubel et.al.\cite{grubel2015performance} has studied the task granularity for a specific problem(1D stencil). Looking at the graph representing the execution time based on the grain size, which resembles a bathtub, we are interested in formulating this graph based on our understanding of the effect of task granularity. Figure~\ref{fig21} shows the execution time in terms of grain size for $DMATDMATADD$ benchmark for matrix size $690\times690$ on $4$ cores.

\vspace{\baselineskip}	
\begin{figure}[H]
	\centering
	\subfloat[]{\includegraphics[scale=.5]{images/bathtub/all_690_4.png}\label{fig20:a}}
	\subfloat[]{\includegraphics[scale=.5]{images/bathtub/tasks_all_690_4.png}\label{fig20:b}}
	\caption{(a)The execution time vs. grain size graph, and (b) execution time vs. number fo tasks graph for $DMATDMATADD$ benchmark for matrix size $690\times690$ ran on $4$ cores.}	
	\label{fig21}
\end{figure}

\vspace{\baselineskip}	
For the sake of simplicity, we change the x axis from grain size to number of tasks. Each specific grain size would create a specific number of tasks(HPX threads), since the parameters we are interested in are directly associated with he number of tasks, we represent execution time based on number of tasks, as shown in Figure~\ref{fig20:b}.
 

Looking at the left hand side of the graph in Figure~\ref{fig20:b}, we can observe that for the first three points, the number of tasks created is smaller than the number of cores(which is 4 in this example). This means that in any of these cases there is at least one idle core, while the other cores are assigned a rather big chunk of work. The performance degradation we observe in that points is associated with starvation, meaning that we are not utilizing our computation resources to the full extent. In these three points, the number of cores actually doing the work is equal to the number of the tasks, since each core gets to execute at most one task.    

To generalize the problem, assuming we are running our application on $N$ cores, with a grain size equal to $g$, $n_t$ tasks are being created, and $M$ cores are actually doing the work. If $n_t<N$, $M$ would be equal to $n_t$, otherwise $M=N$.


From overhead point of view though, if we represent the overhead of creating one task on a particular machine as $\alpha$, the overhead of creating $d$ tasks would be $n_t\alpha$, but this overhead is divided between the $M$ cores actually doing the work. 

To summarize, knowing the grain size, we are expecting the execution time in a many-task runtime system to be mainly affected by these factors, the overhead of creating one task($\alpha$), the number of cores that are actually doing the work($M$), the sequential execution time($t_s$), and finally the portion of the program that could actually be parallelized($\gamma$). 

If we try to integrate these information into a formula, we would expect the relation between execution time($t$) and number of tasks($n_t$) as follows:
\begin{equation}\label{new}
t=\frac{\alpha{n_t}+t_s}{M}+\gamma
\end{equation}

which could be decomposed into these two equations:

\begin{equation}
t=\left\{
\begin{aligned}
\alpha+\frac{t_s}{n_t}+\gamma  \:\:\:\:\:\:\:\:      \text{ if } n_t<N\\
\frac{\alpha{n_t}+t_s}{N}+\gamma\:\:\:\:\:\:\:\:     \text{otherwise}
\end{aligned}
\right.
\end{equation}

Now we use this function to find the best three parameters $\alpha$, $t_s$, and $\gamma$ so that the collected data would fit this model. For this purpose we used the \textit{curve\textunderscore{fit}} package from \textit{SciPy} library in \textit{python}.

In order to make Equation~\ref{new} differentiable, we used the softplus function(Equation~\ref{softplus}) to represent $M$ based on $n_t$.
\begin{equation}\label{softplus}
f(x)=Ln(1+e^x)
\end{equation}

Which results in Equation~\ref{soft_new}:
\begin{equation}\label{soft_new}
t=\frac{\alpha{n_t}+t_s}{(N-1)-Ln(1+(e^{N-1}-1)e^{-n_t})}+\gamma
\end{equation}

Here again we limited our problem to one specific matrix size at a time, and divided the whole data for each matrix size and number of cores into two parts, $60\%$ for training and $40\%$ for testing.


\vspace{\baselineskip}	
\begin{figure}[H]
	\centering
	\subfloat[]{\includegraphics[scale=.45]{images/bathtub/pred/pred_690_4.png}\label{fig22:a}}
	\subfloat[]{\includegraphics[scale=.45]{images/bathtub/pred/pred_690_8.png}\label{fig22:b}}	
	\caption{The prediction of execution time based on grain size using the bathtub model, for (a)4 cores and (b)8 cores for $DMATDMATADD$ benchmark for matrix size $690\times690$.}	
	\label{fig22}
\end{figure}
\vspace{\baselineskip}	
Figure~\ref{fig23} represents the relative error calculated for both training and test set, which is less than $5\%$.

\vspace{\baselineskip}	
\begin{figure}[H]
	\centering
	{\includegraphics[scale=.45]{images/bathtub/error_690.png}}	
	\caption{The error in fitting execution time with the bathtub formula for $DMATDMATADD$ benchmark for matrix size $690\times690$ with different number of cores.}	
	\label{fig23}
\end{figure}

\vspace{\baselineskip}	
Assuming that this function fits our data in acceptable manner, next step would be to check how these three parameters change with the number of cores.

\begin{equation}\label{usl_fit}
f(x)=\frac{m_0}{x}+\frac{m_1(x-1)}{x}+m_2(x-1)+m_3(x)(x-1)
\end{equation}


\vspace{\baselineskip}	
\begin{figure}[H]
	\centering
	\subfloat[]{\includegraphics[scale=.3]{images/bathtub/coef_1_690.png}\label{fig24:a}}
	\subfloat[]{\includegraphics[scale=.3]{images/bathtub/coef_2_690.png}\label{fig24:b}}	
	\subfloat[]{\includegraphics[scale=.3]{images/bathtub/coef_3_690.png}\label{fig24:c}}
	\caption{Fitting the three parameters (a)$\alpha$, (b)$t_s$, and (c)$\gamma$ for $DMATDMATADD$ benchmark for matrix size $690\times690$.}	
	\label{fig24}
\end{figure}


\vspace{\baselineskip}	
We can integrate Equation~\ref{new} and Equation~\ref{usl_fit} to predict the execution time for a given matrix size and number of cores.
For each matrix size having found the parameters $m_0$ to $m_3$, we can find $\alpha$, $t_s$, and $\gamma$ for the particular number of cores requested through Equation~\ref{usl_fit}. Then we can plug in the calculated values for $\alpha$, $t_s$, and $\gamma$ into Equation~\ref{new} to predict the execution time.

Figure~\ref{fig25} shows the prediction error on the test set for $DMATDMATADD$ benchmark for matrix size $690\times690$. The axis shows the different samples in the test set, and the label of each point represents the number of tasks created for that particular data point. As it could be seen, certain number of tasks result in higher prediction error, which needs to be studied.


\vspace{\baselineskip}	
\begin{figure}[H]
	\centering	{\hfill\includegraphics[scale=.35]{images/bathtub/prediction_error_overall690.png}}	
	\caption{The error in fitting execution time with the bathtub formula for $DMATDMATADD$ benchmark for matrix size $690\times690$ with different number of cores.}	
	\label{fig25}
\end{figure}

The problem with the current model is that with this formula we know that the minimum occurs at $n_t=N$, but that's not usually the case. This inspires us to check for a missing factor. 
This model still needs to be studied. The estimate that we have is for execution time, which is in our experiments very small. Changing the graphs from execution time to throughput can help us find the location of the maximum throughput easier compared to minimum of the execution time.


\subsection{Studying the effect of matrix size}


\pagebreak
\singlespacing

%\chapter{Proposed Study}\label{Future}
%\doublespacing
%\section{splittable Task}

In the previous chapter we proposed an analytical model to estimate the execution time of a balanced parallel for loop in terms of the grain size. Based on this model, we offered an approach to find the range of grain size to achieve minimum execution time. The parameters of the proposed model are identified through a benchmark and are exposed to the Blaze library to predict the range of grain size for minimum execution time of a problem at run-time. So the proposed method is a combination of compile-time and run-time solution to improve the performance.\\
In this chapter we choose another direction and look into a run-time adaptive solution to control task granularity in order to achieve the minimum execution time. Why? unbalanced work load.

Utilizing splittable tasks is a runtime adaptive method for managing task granularity, to avoid the large overhead of creating and managing too many tasks due to the fine grain parallelism on one hand, and the starvation resulted from creating less tasks than the available parallelism on the other hand.\\ 
Splittable tasks are tasks that could be partitioned into smaller tasks, when sufficient parallelism is available~\cite{prell2016embracing}. 

~\cite{robison2008optimization}

Prell intensively studies using the splittable tasks for runtime adaptivity in ~\cite{prell2016embracing}. They start by offering steal-half strategy for work stealing in order to steal half of the tasks from a worker's queue at each steal attempt instead of just one task. This could help to avoid creating too much work stealing overhead. But depending on the application, steal-one or steal-half might be the preferable method for work stealing. They propose an adaptive method to decide on whether to use steal-one or steal-half strategy at runtime, based on the current selected strategy and the ratio of the number of executed tasks($M$) within the last $N$ steals, where $N$ is considered the evaluation interval~\cite{prell2016embracing}.\\ 
Next, basing on lazy task scheduling~\cite{tzannes2014lazy}, they suggest to instead of creating at the tasks and decide on how to the workers should steal them, we can create one task and let it split if needed. This way you wouldn't have to deal with the overhead of scheduling tasks along with the overhead of stealing the tasks.\\
At each split, a portion of the task would be added to the current worker's deque as a splittble task to be stolen by free workers, while the rest of the task would be executed by the current worker.    
They propose two splitting strategies namely, and adaptive. 


Our work here is based on Prell~\cite{prell2016embracing}'s definition of splittable tasks and their split strategies. We have implemented an executor within HPX which would create splittable tasks to execute the work, and we improved upon their work in following ways:

\begin{itemize}
\item At each split, instead of allowing the created splittable task to be stolen by free workers, we turn work stealing off, identify the idle workers and explicitly assign the work to one of them. This way we avoid the work stealing overhead originated from unsuccessful steal attempts. 
\item We suggest to utilize our proposed method for identifying the range of grain size for minimum execution time, and use the lower-bound of the range as a cutoff value to stop splitting in order to avoid creating too fine tasks.   
\item We used Apex, an Autonomic Performance Environment for Exascale~\cite{huck2015autonomic}, as a profiling tool to study how tasks are being scheduled.
\end{itemize}




\subsection{Implementation}
For a parallel for-loop with the range of $[a,b)$, one splittable task containing all the iterations from $a$ to $(b-1)$ would be created. Depending on the splitting strategy when a certain condition is met this task would be split into two tasks, $task_1$ containing iterations in the range of $[a,c)$ and $task_2$ would contain iterations from $[c,b)$, where $c$ is the split point. $task_1$ would be executed by the current worker while $task_2$, which is also a splittable task, would be scheduled to run on another core. This allows the runtime to adaptively decide whether to split the current task into smaller tasks or just run it.     

 
\subsection{Splitting Strategies}
The splittable task could be split either based on total number of cores available, or number of idle cores at the time of split. 

\subsubsection{Splitting based on total number of cores}

\subsubsection{Splitting based on number of idle cores}

\subsection{Results}

\subsubsection{For-loop Benchmark}

\begin{figure}[H]
	\centering
	\subfloat[]
	{\centering\includegraphics[scale=.35]{images/hpx_for_loop/splittable/all_idle_cores/marvin_10000_1.png}	
		\label{fig66:a}}
	\subfloat[]
	{\centering\includegraphics[scale=.35]{images/hpx_for_loop/splittable/all_idle_cores/marvin_10000_2.png}	
		\label{fig66:b}}\hfill
	\subfloat[]
	{\centering\includegraphics[scale=.35]{images/hpx_for_loop/splittable/all_idle_cores/marvin_10000_3.png}	
		\label{fig66:c}}
	\subfloat[]
	{\centering\includegraphics[scale=.35]{images/hpx_for_loop/splittable/all_idle_cores/marvin_10000_4.png}	
		\label{fig66:d}}\hfill
	\subfloat[]
	{\centering\includegraphics[scale=.35]{images/hpx_for_loop/splittable/all_idle_cores/marvin_10000_5.png}	
		\label{fig66:e}}
	\subfloat[]
	{\centering\includegraphics[scale=.35]{images/hpx_for_loop/splittable/all_idle_cores/marvin_10000_6.png}	
		\label{fig66:f}}\hfill
	\subfloat[]
	{\centering\includegraphics[scale=.35]{images/hpx_for_loop/splittable/all_idle_cores/marvin_10000_7.png}	
		\label{fig66:g}}
	\subfloat[]
	{\centering\includegraphics[scale=.35]{images/hpx_for_loop/splittable/all_idle_cores/marvin_10000_8.png}	
		\label{fig66:h}}\hfill
	\caption{The results of running the hpx for loop using splittable tasks with all-cores and idle-cores split types compared with different grain sizes, for $problem\_size=10000$, for (a) 1 core, (b) 2 cores, (c) 3 cores, (d) 4 cores, (e) 5 cores, (f) 6 cores, (g) 7 cores, (h) 8 cores. The unit for execution time is microseconds.}
	\label{fig66}	
\end{figure}

\subsubsection{Blazemark}

\subsubsection{Apex}
%\pagebreak
%\singlespacing

%\chapter{Appendix}
%\doublespacing
%\section{Bathtub model}
In the previous section we studied the possibility of using a polynomial to capture the relationship between grain size, number of cores, and throughput for a fixed matrix size, with the purpose of finding a range of grain size that leads us to maximum performance. 
Although the polynomial function was helpful in directing us toward our objective, it does not have a physical implication. 

This motivated us to change our view, and instead of looking just at the data and trying to find a function to fit the data, study the behavior of the data, and then find a function that would be likely to fit the data. That function would be a good fit mostly because that's how we expect the throughput to change with grain size, and not just how the data looks like.   

As stated in Section~\ref{task}, Grubel et.al.\cite{grubel2015performance} has studied the task granularity for a specific problem(1D stencil). Looking at the graph representing the execution time based on the grain size, which resembles a bathtub, we are interested in formulating this graph based on our understanding of the effect of task granularity. Figure~\ref{fig21} shows the execution time in terms of grain size for $DMATDMATADD$ benchmark for matrix size $690\times690$ on $4$ cores.

\vspace{\baselineskip}	
\begin{figure}[H]
	\centering
	\subfloat[]{\includegraphics[scale=.5]{images/bathtub/all_690_4.png}\label{fig20:a}}
	\subfloat[]{\includegraphics[scale=.5]{images/bathtub/tasks_all_690_4.png}\label{fig20:b}}
	\caption{(a)The execution time vs. grain size graph, and (b) execution time vs. number fo tasks graph for $DMATDMATADD$ benchmark for matrix size $690\times690$ ran on $4$ cores.}	
	\label{fig21}
\end{figure}

\vspace{\baselineskip}	
For the sake of simplicity, we change the x axis from grain size to number of tasks. Each specific grain size would create a specific number of tasks(HPX threads), since the parameters we are interested in are directly associated with he number of tasks, we represent execution time based on number of tasks, as shown in Figure~\ref{fig20:b}.
 

Looking at the left hand side of the graph in Figure~\ref{fig20:b}, we can observe that for the first three points, the number of tasks created is smaller than the number of cores(which is 4 in this example). This means that in any of these cases there is at least one idle core, while the other cores are assigned a rather big chunk of work. The performance degradation we observe in that points is associated with starvation, meaning that we are not utilizing our computation resources to the full extent. In these three points, the number of cores actually doing the work is equal to the number of the tasks, since each core gets to execute at most one task.    

To generalize the problem, assuming we are running our application on $N$ cores, with a grain size equal to $g$, $n_t$ tasks are being created, and $M$ cores are actually doing the work. If $n_t<N$, $M$ would be equal to $n_t$, otherwise $M=N$.


From overhead point of view though, if we represent the overhead of creating one task on a particular machine as $\alpha$, the overhead of creating $d$ tasks would be $n_t\alpha$, but this overhead is divided between the $M$ cores actually doing the work. 

To summarize, knowing the grain size, we are expecting the execution time in a many-task runtime system to be mainly affected by these factors, the overhead of creating one task($\alpha$), the number of cores that are actually doing the work($M$), the sequential execution time($t_s$), and finally the portion of the program that could actually be parallelized($\gamma$). 

If we try to integrate these information into a formula, we would expect the relation between execution time($t$) and number of tasks($n_t$) as follows:
\begin{equation}\label{new}
t=\frac{\alpha{n_t}+t_s}{M}+\gamma
\end{equation}

which could be decomposed into these two equations:

\begin{equation}
t=\left\{
\begin{aligned}
\alpha+\frac{t_s}{n_t}+\gamma  \:\:\:\:\:\:\:\:      \text{ if } n_t<N\\
\frac{\alpha{n_t}+t_s}{N}+\gamma\:\:\:\:\:\:\:\:     \text{otherwise}
\end{aligned}
\right.
\end{equation}

Now we use this function to find the best three parameters $\alpha$, $t_s$, and $\gamma$ so that the collected data would fit this model. For this purpose we used the \textit{curve\textunderscore{fit}} package from \textit{SciPy} library in \textit{python}.

In order to make Equation~\ref{new} differentiable, we used the softplus function(Equation~\ref{softplus}) to represent $M$ based on $n_t$.
\begin{equation}\label{softplus}
f(x)=Ln(1+e^x)
\end{equation}

Which results in Equation~\ref{soft_new}:
\begin{equation}\label{soft_new}
t=\frac{\alpha{n_t}+t_s}{(N-1)-Ln(1+(e^{N-1}-1)e^{-n_t})}+\gamma
\end{equation}

Here again we limited our problem to one specific matrix size at a time, and divided the whole data for each matrix size and number of cores into two parts, $60\%$ for training and $40\%$ for testing.


\vspace{\baselineskip}	
\begin{figure}[H]
	\centering
	\subfloat[]{\includegraphics[scale=.45]{images/bathtub/pred/pred_690_4.png}\label{fig22:a}}
	\subfloat[]{\includegraphics[scale=.45]{images/bathtub/pred/pred_690_8.png}\label{fig22:b}}	
	\caption{The prediction of execution time based on grain size using the bathtub model, for (a)4 cores and (b)8 cores for $DMATDMATADD$ benchmark for matrix size $690\times690$.}	
	\label{fig22}
\end{figure}
\vspace{\baselineskip}	
Figure~\ref{fig23} represents the relative error calculated for both training and test set, which is less than $5\%$.

\vspace{\baselineskip}	
\begin{figure}[H]
	\centering
	{\includegraphics[scale=.45]{images/bathtub/error_690.png}}	
	\caption{The error in fitting execution time with the bathtub formula for $DMATDMATADD$ benchmark for matrix size $690\times690$ with different number of cores.}	
	\label{fig23}
\end{figure}

\vspace{\baselineskip}	
Assuming that this function fits our data in acceptable manner, next step would be to check how these three parameters change with the number of cores.

\begin{equation}\label{usl_fit}
f(x)=\frac{m_0}{x}+\frac{m_1(x-1)}{x}+m_2(x-1)+m_3(x)(x-1)
\end{equation}


\vspace{\baselineskip}	
\begin{figure}[H]
	\centering
	\subfloat[]{\includegraphics[scale=.3]{images/bathtub/coef_1_690.png}\label{fig24:a}}
	\subfloat[]{\includegraphics[scale=.3]{images/bathtub/coef_2_690.png}\label{fig24:b}}	
	\subfloat[]{\includegraphics[scale=.3]{images/bathtub/coef_3_690.png}\label{fig24:c}}
	\caption{Fitting the three parameters (a)$\alpha$, (b)$t_s$, and (c)$\gamma$ for $DMATDMATADD$ benchmark for matrix size $690\times690$.}	
	\label{fig24}
\end{figure}


\vspace{\baselineskip}	
We can integrate Equation~\ref{new} and Equation~\ref{usl_fit} to predict the execution time for a given matrix size and number of cores.
For each matrix size having found the parameters $m_0$ to $m_3$, we can find $\alpha$, $t_s$, and $\gamma$ for the particular number of cores requested through Equation~\ref{usl_fit}. Then we can plug in the calculated values for $\alpha$, $t_s$, and $\gamma$ into Equation~\ref{new} to predict the execution time.

Figure~\ref{fig25} shows the prediction error on the test set for $DMATDMATADD$ benchmark for matrix size $690\times690$. The axis shows the different samples in the test set, and the label of each point represents the number of tasks created for that particular data point. As it could be seen, certain number of tasks result in higher prediction error, which needs to be studied.


\vspace{\baselineskip}	
\begin{figure}[H]
	\centering	{\hfill\includegraphics[scale=.35]{images/bathtub/prediction_error_overall690.png}}	
	\caption{The error in fitting execution time with the bathtub formula for $DMATDMATADD$ benchmark for matrix size $690\times690$ with different number of cores.}	
	\label{fig25}
\end{figure}

The problem with the current model is that with this formula we know that the minimum occurs at $n_t=N$, but that's not usually the case. This inspires us to check for a missing factor. 
This model still needs to be studied. The estimate that we have is for execution time, which is in our experiments very small. Changing the graphs from execution time to throughput can help us find the location of the maximum throughput easier compared to minimum of the execution time.


\subsection{Studying the effect of matrix size}


%\pagebreak
%\singlespacing

%To insert additional chapters, copy the previous five lines, using chapterX as the argument of the 
%\input command for Chapter X, where X=6,7,8,...
\addtocontents{toc}{\vspace{12pt}}
\addcontentsline{toc}{chapter}{\hspace{-1.6em} REFERENCES}
%\begin{thebibliography}{999}
\vspace{0.9em}
\bibliographystyle{unsrt}
\bibliography{bibliography}
%\input{bibliography}
%\end{thebibliography}


\addtocontents{toc}{\vspace{12pt} \hspace{-1.8em} APPENDIX \vspace{-1em}}
\pagebreak
\singlespacing
\appendix
%\chapter{Copyright Information}
%\vspace{0.5em}
%\input{appendixA}
%\pagebreak
%\addtocontents{toc}{\vspace{12pt} \hspace{-1.8em} APPENDIX \vspace{-1em}}
%\appendix
%\chapter{Title of Appendix A}
%\vspace{0.5em}
%\input{appendixA}
%\pagebreak
%\chapter{Title of Appendix B}
%\vspace{0.5em}
%\input{appendixB}
%\pagebreak
%If you need to insert additional appendices, copy the previous four lines, using appendixY as the
%argument of the \input commnd for Appendix Y, for Y=C,D,E,...

%Finally, the vita section is created and included in the Table of Contents.
%\chapter*{Vita}
%\doublespacing
%\setlength{\parindent}{1.75em}
%\vspace{0.2em}
%\addtocontents{toc}{\vspace{12pt}}
%\addcontentsline{toc}{chapter}{\hspace{-1.5em} VITA}
%Insert the text of your vita, which is basically a description of yourself and your academic career.
%\singlespacing
%\chapter*{Vita}
%\doublespacing
%\setlength{\parindent}{1.75em}
%\vspace{0.2em}
%\addtocontents{toc}{\vspace{12pt}}
%\addcontentsline{toc}{chapter}{\hspace{-1.5em} VITA}
%VITA
\end{document}





