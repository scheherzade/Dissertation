\section{Literature Review}

Loop scheduling techniques has been extensively studied by different researchers. In \cite{donfack2012hybrid} the authors propose a hybrid static/dynamic method for loop scheduling that improves the performance of dense matrix factorization, compared to both fully static and fully dynamic scheduling. The authors of \cite{donfack2012hybrid}, divide the dependency graph into two subgraphs, one of which is scheduled dynamically and the other one is scheduled statically. The tasks on the critical path are scheduled statically and each thread is forced to prioritize the static tasks\cite{donfack2012hybrid}. They were able to improve data locality and scheduling overhead, while creating a more balanced workload. 

\cite{xue2007locality}
 \cite{tang1986processor},\cite{polychronopoulos1987guided},\cite{hummel1992factoring},\cite{kruskal1985allocating}
	




The previous work on predicting the performance of a parallel application mainly focuses on three major types of models: analytical, trace-based, and empirical models\cite{malakar2018benchmarking}. 

The analytical models\cite{blagojevic2008modeling},\cite{kerbyson2001predictive},\cite{valiant1990bridging}, while providing an arithmetic formula to represent the execution time of an application, require a deep understanding of the application, to apply platform-specific optimizations, and can not be generalized to different domains and architectures\cite{lee2007methods},\cite{sun2017automated},\cite{pllana2007performance}.
Traced-based models, on the other hand, use the traces collected through instrumentation, to predict the performance. These models, opposed to analytical models, do not rely on an expert's knowledge of the application, but while adding some overhead to the runtime, these models require a large storage space to save the traces, and are hard to interpret\cite{sun2017automated}.   
In empirical modeling, the results obtained from running an application with a set of parameters on a specific set of machines to build a model for unknown set of application and system parameters\cite{malakar2018benchmarking}. This type of modeling includes machine learning based approaches.

In \cite{ipek2005approach}, the authors use neural networks to predict the performance focusing on SMG2000 application, a parallel multigrid solver for linear systems\cite{falgout2002hypre}, on two different platforms. Defining application parameters $N_x$, $N_y$, $N_z$, representing the working set size per processor, and $P_x$, $P_y$, $P_z$, describing the three-dimension processor topology, as the features, \cite{ipek2005approach} uses a fully connected neural network to learn the model. Since they use absolute mean square error as the loss function, they use stratification to replicate samples with lower values by a factor which is proportional to their target value. They also apply bagging technique to decrease the variance in the model. As they increase the size of the training set to 5K points, they reach an error rate of 4.9\%. 

As a trace-based model, \cite{sun2017automated} analyzes the abstract syntax tree of the code and collects data through inserting special code for instrumentation when encounters 4 different situations, namely, assignments, branches, loops, and MPI communications. The authors then use 5 different machine learning methods including random forests, support vector machine, and ridge regression to build a prediction model from the collected data. Through applying two filtration processes, they were able to decrease the amount of overhead introduced along with the storage space requirement. Their results were inclined towards random forest, mainly because of the lower impact of categorical features on it, which is helpful in general cases where we do not have any knowledge about the type of features\cite{sun2017automated}.  
	
In \cite{malakar2018benchmarking} the authors investigate a set of machine learning techniques, including deep neural networks, support vector machine, decision tree, random forest, and k-nearest neighbor to predict the execution time of 4 different applications. Each of these applications require a certain set of features as input, for example, for the miniMD application in molecular dynamics, the number of processes and the number of atoms were considered as the input features, while for miniAMR, an application for studying adaptive mesh refinement, number of processes and also block sizes in $x$, $y$, and $z$ direction, where used as the input features. While achieving promising results especially for deep neural networks, bagging, and boosting methods, \cite{malakar2018benchmarking} suggest utilizing transfer learning through deep neural networks to predict performance on other platforms.
	
	
\cite{pusukuri2011thread}
\cite{marin2004cross}


Although concentrating on GPUs,	\cite{liu2018runtime} proposes a lightweight machine learning based performance model to choose the number of threads to use for parallelization for a specific data size and operation. With the final goal of improving the training time in a neural network, \cite{liu2018runtime} selects 4 performance features collected by hardware counters namely, number of CPU cycles, number of cache misses, cache accesses for the last cache level, and number of level 1 cache hits. Then they take two different approaches to build their model. In the first on they try 10 different regression models including random forest, and in the second one they use hill climbing algorithm to choose the number of threads. In addition to hardware independent, and not requiring the training process, hill climbing algorithm achieves a much higher accuracy compared to the best performing regression model.

In this paper, we suggest using machine learning to directly predict the optimal chunk size to achieve the best performance instead of predicting the execution time or the optimal number of cores to run the application on. For this purpose, we have offered a set of general features that are not specific to an application and could easily be extracted at compile time or at run time. Once the data has been collected and our model has been created, the prediction results could be easily applied to a new application with a negligible overhead. 


\cite{sun2017automated}


	
As another field to use machine learning, \cite{qawasmeh2015adaptive} collects seven runtime events and uses machine learning not to predict the performance, but to schedule the tasks. These events include, task creation, suspension, execution, completion, implicit/explicit barrier, parallel region, and finally loop/master/single region runtime events, collected through the OMPT using ORA API. Experimenting with four different machine learning techniques, including support vector machine, random forest, neural networks, and naive bayes, they would select one specific task pool configuration out of the three pre-defined options as the final classification result. Testing this framework on a real life molecular dynamics application, they observed an up to 31\% improvement in performance. 


Compiler-based methods:

The authors of \cite{wang2009mapping} propose using machine learning to predict the optimal number of threads, and also the optimal scheduling policy for running an OpenMP application. Through that, they were able to develop an automatic compiler-based method to map a parallel application to a multicore processor. They collect three type of features namely, code, data, and runtime features. Code features are extracted from the code directly, and they include cycles per instruction, number of branches, load and store instructions, and computations per instruction. While the code features could be collected statically at compile time, the data and run-time features are collected through low-cost profiling runs. This group of features include loop iteration count, branch miss rate, and $L1$ data cache miss rate. The authors of \cite{wang2009mapping} then use an artificial neural network to predict the speedup achieved for a program with certain number of threads, and at the same time they use a support vector machine model to predict the best scheduling policy, out of block, cyclic, dynamic, and guided scheduling policies, for an unseen program.



	
	
	
	
\cite{treibig2012performance} 
profiling information about the application on a given architecture
\cite{cammarota2012just},\cite{zhang2005runtime},\cite{thoman2012automatic}

Machine learning models 
\cite{singh2009real}, \cite{zomaya2001observations}, \cite{qawasmeh2015adaptive}



	

	

\cite{li2009machine}
\vspace{\baselineskip}


